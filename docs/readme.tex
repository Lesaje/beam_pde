% Delayed Timoshenko Beam Transmission Solver Documentation
\documentclass[a4paper,11pt]{article}
\usepackage{amsmath,amssymb,geometry,graphicx,hyperref}
\geometry{margin=1in}
\title{Solver for Delayed Timoshenko Beam Transmission Problem}
\author{}
\date{}

\begin{document}
\maketitle

\section*{1. Problem Formulation}
We solve a Timoshenko beam transmission problem on two subdomains
\(I_1=(0,L_0)\) and \(I_2=(L_0,L)\), with transverse displacement \(\phi_i(x,t)\) and shear angle \(\psi_i(x,t)\), \(i=1,2\), together with an auxiliary transport–delay variable \(z(x,t)\) on \(I_1\).

\subsection*{1.1 Governing PDEs on \(I_i\)}
For \(i=1,2\), and \(x\in I_i\),
\begin{align}
\rho_i\,\phi_{i,tt} & - \kappa_i\,(\phi_{i,x}+\psi_i)_x + f_1(\phi_i,\psi_i) = g_i(x), \\
\sigma_i\,\psi_{i,tt} & - \lambda_i\,\psi_{i,xx} + \kappa_i\,(\phi_{i,x}+\psi_i) + f_2(\phi_i,\psi_i) = h_i(x).
\end{align}
The nonlinear terms are
\[
f_1(\phi,\psi) = 4(\phi+\psi)^3 - 2(\phi+\psi) + 2\phi\,\psi^2,
\qquad
f_2(\phi,\psi) = 4(\phi+\psi)^3 - 2(\phi+\psi) + 2\phi^2\,\psi.
\]
The (time‐independent) load functions are
\[
g_1(x)=\sin x, \; h_1(x)=x, \quad
g_2(x)=\cos x, \; h_2(x)=x+1.
\]

\subsection*{1.2 Transport–Delay PDE on \(I_1\)}
The auxiliary variable \(z(x,t)\) satisfies
\[
\tau\,z_t + z_x = 0, \quad x\in[0,L_0].
\]
Its boundary values couple to \(\psi_1\):
\[
z(0,t) = \psi_{1,t}(0,t),
\qquad
z(L_0,t) = \psi_{1,t}\bigl(0,t-\tau L_0\bigr).
\]

\subsection*{1.3 Transmission Conditions at \(x=L_0\)}
Continuity of fields:
\[
\phi_1(L_0,t)=\phi_2(L_0,t),
\quad
\psi_1(L_0,t)=\psi_2(L_0,t).
\]
Continuity of fluxes:
\[
\kappa_1(\phi_{1,x}+\psi_1)=\kappa_2(\phi_{2,x}+\psi_2),
\quad
\lambda_1\psi_{1,x}=\lambda_2\psi_{2,x},
\quad x=L_0.
\]

\subsection*{1.4 Boundary Conditions with Delay}
At \(x=0\) (left end of \(I_1\)):
\[
\phi_1(0,t)=0,
\quad
\lambda_1\psi_{1,x}(0,t) = \gamma\,z(0,t) + \mu\,z(L_0,t).
\]
At \(x=L\) (right end of \(I_2\)):
\[
\phi_2(L,t)=0,
\quad
\psi_2(0,t)=0.
\]

\subsection*{1.5 Initial Conditions}
For \(x\) in each subdomain,
\begin{align*}
&\phi_i(x,0)=\phi_i^0(x), \quad \phi_{i,t}(x,0)=\phi_i^1(x),\\
&\psi_i(x,0)=\psi_i^0(x), \quad \psi_{i,t}(x,0)=\psi_i^1(x),\\
&z(x,0)=z_0(x)=\tfrac{13}{4}x+1.
\end{align*}

\subsection*{1.6 Parameter Values}
The constants are set to:
\[
\gamma = 1,
\,\rho_1=1,\rho_2=2,
\,\sigma_1=2,\sigma_2=1,
\,\kappa_1=4,\kappa_2=1,
\,\lambda_1=8,\lambda_2=4,
\,L_0=4,\,L=10,
\,\mu=\tfrac12,\,\tau=1.
\]

\section*{2. Code Overview}
Each code block is documented with its purpose.

\subsection*{2.1 Imports and Damping}
\begin{verbatim}
import numpy as np
import matplotlib.pyplot as plt
from mpl_toolkits.mplot3d import Axes3D

# small linear damping to stabilize
beta_phi1,beta_psi1=0.01,0.01
beta_phi2,beta_psi2=0.01,0.01
\end{verbatim}
\noindent
Sets up array operations, plotting, and small artificial damping coefficients.

\subsection*{2.2 Parameter and Grid Setup}
\begin{verbatim}
# Physical & delay PDE parameters
gamma=1.0;rho1,rho2=1.0,2.0;sigma1,sigma2=2.0,1.0
kappa1,kappa2=4.0,1.0;lambda1,lambda2=8.0,4.0
L0,L=4.0,10.0;mu=0.5;tau_delay=1.0

# Discretization
T=10.0;n1,n2=50,50;m=2000
dt=T/(m-1);h1=L0/(n1-1);h2=(L-L0)/(n2-1)

t=np.linspace(0,T,m)
x1=np.linspace(0,L0,n1)
x2=np.linspace(L0,L,n2)
delay_steps=max(1,int(tau_delay*L0/dt))
\end{verbatim}

\subsection*{2.3 Solution Arrays Allocation}
\begin{verbatim}
phi1=np.zeros((n1,m));psi1=np.zeros((n1,m))
phi2=np.zeros((n2,m));psi2=np.zeros((n2,m))
z   =np.zeros((n1,m))
\end{verbatim}
Preallocate fields on each domain and the transport variable \(z\).

\subsection*{2.4 Loads and Nonlinearities}
\begin{verbatim}
g1=lambda x:np.sin(x)
h1_fun=lambda x:x
g2=lambda x:np.cos(x)
h2_fun=lambda x:x+1
f1=lambda phi,psi:4*(phi+psi)**3-2*(phi+psi)+2*phi*psi**2
f2=lambda phi,psi:4*(phi+psi)**3-2*(phi+psi)+2*phi**2*psi
\end{verbatim}
Defines external loads and the cubic nonlinear terms.

\subsection*{2.5 Initial Conditions}
\begin{verbatim}
# phi1^0,psi1^0 on [0,L0]
phi1[:,0]=-19/16*x1**2+200/32*x1
psi1[:,0]=x1
# phi2^0 continuous ramp; psi2^0 given
phi2[:,0]=phi1[-1,0]*(L-x2)/(L-L0)
psi2[:,0]=x2**3-(166/9)*x2**2+(914/9)*x2-1540/9

# First time step from phi_i^1,psi_i^1
dphi1_0=x1
psi1_0=10-x1
dphi2_0=(2/3)*(10-x2)
dpsi2_0=(5/6)*(10-x2)
phi1[:,1]=phi1[:,0]+dt*dphi1_0
psi1[:,1]=psi1[:,0]+dt*psi1_0
phi2[:,1]=phi2[:,0]+dt*dphi2_0
psi2[:,1]=psi2[:,0]+dt*dpsi2_0

# z(x,0) and first upwind step
z[:,0]=13/4*x1+1
for j in range(1,n1):
    z[j,1]=z[j,0]-dt/(tau_delay*h1)*(z[j,0]-z[j-1,0])
\end{verbatim}

\subsection*{2.6 Time‐Stepping Loop}
Pseudocode structure:
\begin{enumerate}
  \item Update interior of \(I_1\) and \(I_2\) with 2nd‑order finite differences, explicit in time, plus damping.
  \item Upwind transport update for \(z\).
  \item Enforce Dirichlet BCs: \(\phi_1(0)=0\), \(\phi_2(L)=0\), \(\psi_2(0)=0\).
  \item Transmission at \(x=L_0\): average adjacent nodes to enforce continuity.
  \item Delayed‐feedback BC for \(\psi_1(0)\): compute and clamp derivatives, then solve for boundary value.
\end{enumerate}

\subsection*{2.7 Visualization}
\begin{verbatim}
X=np.concatenate((x1,x2))
Phi=np.vstack((phi1,phi2))
fig=plt.figure()
ax=fig.add_subplot(111,projection='3d')
Tm,Xm=np.meshgrid(t,X)
ax.plot_surface(Tm,Xm,Phi,rstride=10,cstride=10,cmap='viridis')
ax.set(xlabel='t',ylabel='x',zlabel='phi(x,t)')
plt.show()
\end{verbatim}
Plots the concatenated solution surface \(\phi(x,t)\).

\end{document}

